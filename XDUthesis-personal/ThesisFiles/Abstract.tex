% 中文摘要
\begin{abstract}
显微成像的应用开始在越来越多的领域中得到应用。光学薄膜的表面粗糙度对其光学特性的测量精度有较大影响。在常见的加工生产过程中,薄膜表面的毛刺、划痕等都可能对精密产品的性能造成影响。为提高产品的光学特性的精度,对相应产品出厂前表面进行实时的、高精度的缺损情况测试具有实际重要意义。使用电子仪器的手段相比于传统人力目测检验方法,具有单位效率高、判定率高、可靠性高等特点,便于元器件表面缺损测试方面向现代微型组件化智能化发展。
本文依托ARM处理器平台设计了数字化显微成像技术处理系统,并对零部件表面的缺损情况进行了实时测试。此检测系统具有智能化、小型化、图像采集方便实时的特点,可以进一步提高产品出厂前合格率、降低后续产品返修率。整体上,采用索尼的堆栈式CMOS图像传感器IMX135,作为核心数字显微图像采集芯片,借助CSI-2接口将图像传至ARM主板,并利用ARM处理器完成显微图像的图像增强、图像识别等算法处理,并实现了显微图像的后续存储、归档和显示。存储、归档上可以放在存储卡或云端,通过HDMI或网络将图像传输到LED平板显示器上进行显示,达到实现系统集成化、桌面一体化、处理器实时处理的常见功能,同时备选两套方案,即可使用HDMI线缆将显微图像传输到桌面显示,利用以太网或802.11协议,借助路由也可以在异地通过互联网的方式,通过浏览器访问显示显微图像。更好实现产品缺损检测效果的远程实时控制功能。
本论文完成下面工作:


\end{abstract}
\keywords{数字显微成像, 光学薄膜, ARM  }

% 英文摘要
\begin{enabstract}
  This paper is just a sample example for the users in learning the \XDUthesis. I will try my best to use the commands and environments which are involved by the \XDUthesis. Also, the popular composition skills in figures, tables and equations will be elaborated.
  
  In the part unimportant, I will show something others, such as poems and lyrics.

\end{enabstract}
\enkeywords{XDUthesis, commands, environments, skills}