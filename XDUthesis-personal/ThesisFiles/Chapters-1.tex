\chapter{引言}
\section{选题背景及意义}
\subsection{选题研究内容(待改进)}
伴随着技术的革新和工艺上的不断进步,工业产品的表面精度检测要求不断提升,目前的缺陷检测方面技术开始越加复杂,各种的技术的工作方式和原理开始出现变更。本文主要说明了新型技术的使用和传统方式的对比。传统人工方式效率低,出错率较高,借助现有的图像增强和识别技术,可以加快验证进程,进一步匹配流水线。此外,阐述了实时检测的产品识别的价值和意义。

为应对急剧增长的系统使用环境,需要考虑系统的可扩展性,此处主要借助模块的低耦合性完成,主体上使用单一agent,并架设服务器构建公网平台,使用注册机制进行系统管理。

关注光学成像部分,设计光学系统,实现图像的处理,归档,显示,性能报表等相关功能。主要利用平台上的图像相关的API,完成系统的图像处理部分,可包括去噪,模块识别等功能。此外,关于数据的归档,显示和报表可进行额外的分析和处理、综合。硬件选用上,考虑ARM系列市场份额持续上升,学习成本较低,适应于通用型应用,选用基于ARM的嵌入式处理器技术,其相比于FPGA和DSP,有通用性强,占用体积小,能耗低,性价比高,平台兼容性好等多方面特性。CPU为RISC架构,单一指令周期,长流水线,TDP可以做到较低,于是此处通用选型选用ARM作为主控。在成像方面的传感器上,选用手机上常见的摄像头模块,易于获得,接口选用高速串行的CSI接口,频率高,单位数据传输量大,在后续处理中完成图像的预处理,辨识问题图片的编号,全流程检测链的管理,做到单位流水效率提高。

图像系统采集的硬件设计和选型。在硬件设计上,需要考虑成本分配,易用性,可扩展性,兼容性,平台性能,产品参数等多方面因素。此次主要需要借助GPU的图像处理,因此不考虑DSP和FPGA模拟处理和外接芯片的方式,在版型上,为了库移植的方便以及资料的充分情况,决定选用raspberrypi老版本。
\subsection{解决问题(待改进)}
硬件成本的不断下降使曾经昂贵的技术使用更加广泛,显微成像的单一器具已经开始应用在生活的部分方面,在此基础的改进上,增加数据的处理模块,在不同指标不同场合的跟进测试,是目前的一个研究热点。
在此需求上,本文设计了基于树莓派的数字显微成像系统,尝试在缺陷材料上进行进一步测试。在设计中,选用了索尼的第二代BSI光照CMOS图像传感器IMX135,IMX135 有最高60fps,4208x3120的分辨率,小至1umx1um的像元尺寸,总像素为1300万,利用低压差分信号抗干扰来提高信噪比,使图像成像效果更佳。
该显微成像系统是传统嵌入式系统向智能化、互联网化的尝试。旨在解决传统供应链上效率低,人力成本占用太高的问题。借助嵌入式与云计算,进一步提高产品合格率检测流水,提高每日检测样品数和准确率。并在计算设备中进行资料的归档和备份处理。在设计中,关注系统的可扩展性,部署迁移的高效性。
探讨互联网在传统的行业的解决方案,使用技术手段去除过度冗余和低效问题,提高数据的利用率,充分利用机器的效用。远程控制和检测的方式提高员工的流动性,完成更适合人力解决的部分。在图像处理方面,考虑系统的可扩展性和伸缩性的前提,结合PaaS下的图像处理API,完成产品的进一步检测。监控平台可以在互联网上进行访问,亦可直接接入显示器进行产品检验。

\section{国内外研究现状}
\subsection{显微成像国内外发展现状}
显微镜的起源较早,但是在后期才发展快速。早在2500年前的《墨经》中,就已经有了凹透镜的记载,然而凸透镜的发展却一直没有进展。直到16世纪末期,荷兰的眼镜商詹森( Zaccharias Janssen ) 和他的儿子尝试了一次将两个凸透镜放入到同一个镜筒中,发明了人类历史上第一个光学显微镜。而后在1609年,著名的天文学之父伽利略在这个实验的基础上,分解出相关的物理学原理,并依据自身的理论发明了更好聚焦的显微镜。而后的列文虎克将放大倍数大大的提升了一截,第一次能够发现微观的生物和非生物现象,出版了极多的论文,同时极大的推进了生物学的发展。后之显微镜之父,罗伯特﹒虎克仿制了一台与列文虎克一样的显微镜,证实了水中微小微生物的发现,1830年,利斯忒(Joseph Jackson Lister,1786-1869)用几个有特定间距的透镜组减小了球面像差,从而进一步改进了显微镜。而后,1872年德国的数学家和物理学家 Ernst Abbe 改进了玻璃的制造和生产工艺,改进并生产出有均一折射率的光学级玻璃。
受到衍射限制,光学显微镜的分辨率达到显微镜的分辨率极限,当然,这也是光学显微镜能够突破的分辨率极限,即为


近来,,,blabla

在科学研究领域,blabla。

\subsection{光学薄膜检测的发展现状}
起源

商业应用

检测

完善

重要性

\subsection{本文主要任务和安排}
\subsubsection{主要任务和安排}



\subsubsection{具体章节结构}


