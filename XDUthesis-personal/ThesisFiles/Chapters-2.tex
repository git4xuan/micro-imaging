\chapter{显微成像系统硬件设计}
\section{显微成像光学原理}
\subsection{显微成像系统概述}
显微镜成像可以看做是一个两级放大系统,从结构上看是两个光学系统的组合, 其中,对观察物体直接进行尺寸放大的一组透镜成为显微物镜,靠近眼睛、扩大视角 的放大镜是显微目镜。一般的显微物镜和显微目镜的结构都比较复杂,都是由一系列 的透镜组构成。目视显微镜的像面是观察者的眼睛,而数字显微镜的像面是图像传感 器,数字显微镜发展到现在已经有多种多样的形式。但其中最重要的几点在设计中必 须注意到,例如设计中需要更多的考虑视场匹配的问题,即设计的显微光学系统的成 像面与图像传感器的大小匹配、被测物的光源照明光源特性和系统的照明系统相匹配 [19]。数字显微成像系统必须考虑光学镜头部分与 CMOS 传感器的直接的转接与衔接。 
\subsection{显微成像原理}
在图 2.1 中,和分别是显微物镜和目镜,AC 为物镜前一定距离处的物体,人眼在 目镜后进行观察,物镜像方焦点与目镜物方焦点之间的距离,即物镜与目镜组合的光 学间隔称为光学筒长[20]。物镜前方的物体 AC,它的位置处于物镜一倍焦距和两倍物镜 焦距之间,经过物镜后形成为倒立放大的实像,使得处于目镜的焦平面上,目标经过 显微目镜放大为虚像之后,以供给眼睛来观察,显微目镜的物方焦点和之间的距离决 定着虚像的位置,它既可以出现在无限远处,也可以成像在观察者的明视距离处。
\subsection{显微镜分辨率和放大率}
显微镜存在光阑,由于光的衍射效应,一个物点在像面上形成的应该是圆孔的衍 射斑。根据瑞利判据,这时光学系统只能够分辨物面上一定大小的物体,即显微镜系 统存在最小的分辨尺寸。显微镜分辨率通常采用它所能分辨的视场下两物点间最小距 离来表述。由光的衍射理论知,其分辨率表达式为 

\section{显微成像处理平台材料选型}
\subsection{显微镜头物镜目镜选型}
(1)显微物镜的设计 在本系统设计中,对测量精度要求较高,要求系统分辨本领达到 0.5m。本显微成 像系统的分辨率由光学镜头的分辨率和 CMOS 摄像头的分辨率共同决定。系统中光学 镜头部分对物体进行放大,先考虑光学系统的分辨率:
上式中, n 为物空间介质折射率,为照明光源的中心波长,是系统的分辨率,因 此在照明波长确定时,只能增大镜头的数值孔径来提升分辨率。 整个系统光学放大率由系统要求的分辨率大小和CMOS的像元尺寸决定,如下式,
式中,是系统中 CMOS 芯片的最小可分辨尺寸,为系统的分辨率,为系统总的放 大率。 根据上述的设计分析,选用了消色差物镜中放大倍数中等的李斯特物镜。它的结 构采用两个分离的双胶组合透镜组,垂轴放大率达到 10 倍,数值孔径(NA)设计值 为 0.61。李斯特型物镜的设计原则包括:(一)各个双胶合透镜组具有相同的偏角,后 组比前组的偏角略大。(二)光阑位置处于第一个双胶合透镜处。采用两组双胶合透镜 来抵消球差和慧差,物镜的总焦距就等于两组双胶合透镜之间的距离。前一个双胶合 组的焦距两倍于物镜焦距。物镜的总焦距与第二个双胶合组焦距相同。(三)采取两个 双胶合透镜分别单独校正系统的球差、慧差和色差,这种设计的优点是采用两个双胶 合透镜组合,组合在一起为一个中倍显微物镜,移去一个双胶合组后可作为低倍显微 物镜[28]。 按照物镜设计要求:物镜要求放大倍数为 10 倍,系统光源照明采用中心波长为 550nm 的光源,由式 NA=nsinu 计算可得,透镜数值孔径大小为 0.61,通过以上几个参 数的确定,选出符合参数要求的镜片组合。选定透镜组合后采用 ZEMAX 对设计进行 仿真,对光线进行追迹、计算像差,对设计不满意的参数,再次重新选择玻璃材料, 重复上面的仿真计算[29],直到达到设计要求。
\subsection{数字图像传感器功能选型}
作为嵌入式显微成像系统的图像采集的成像核心器件,图像传感器的选择显得至 关重要。目前在数字图像传感领域最重要的两种器件是 CCD 和 CMOS 图像传感器, 这两种图像传感器分别采用硅材料和互补金属氧化物材料制作而成,在成像质量及像 素数据的读出原理方面均存在着不同。作为数字图像的采集就必须对这两种传感器加 以比较,选择最合适的传感器,下面简单介绍下这两种图像传感器。 (一)CCD 与 CMOS 传感器的比较 (1)CCD 图像传感器 CCD 的中文名字是电荷耦合器件(ChargeCoupledDevice),它经过近 40 年历史 的发展,从最初的几十个像素的线型 CCD 发展到现在的几千万像素的大面阵 CCD, 不论在半导体制作工艺还是在成像质量方面,都取得了巨大飞跃,并在科学研究和日 常电子消费品中得到了广泛应用。目前商业化应用的 CCD 图像传感器涵盖了红外、 紫外和可见光等光谱范围内多种 CCD,它的像元阵列与结构也在着线型和面阵型两 大类,面阵 CCD 按读出方式可分为全帧转移、帧内转移、累进扫描、隔列转移等, CCD 芯片中的光敏单元按阵列式排布,各个对光敏感单元都是一个像素单元,外界 光线照射在像素阵列上转换成模拟电信号,再经过模数转换器转为数字图像信号,并 在一定的时序的驱动下按一定的图像格式输出到外部处理器上。 CCD 工作原理是以电荷作为图像信号,实现对电荷的有效存储和电荷的序列化 转移。因此 CCD 的工作过程大致分为:光电转换、电荷存储、电荷转移和电荷输出。 首先,光电转换是根据照射到摄影面上的光强弱产生电荷,也就是硅二极管中的电子 从光子中获得能量后改变状态,只要施加少许电场电子就可以呈现自由运动的状态的 现象。光生电荷存储原理为当外界光照射到光敏面上时,光子穿过上面的电极层和氧 化材料层,投射在到硅基底材料上,硅基底在吸收光子的同时就会激发出电子空穴对, 被硅材料吸收的光子存储到光生电荷反型区,证实了 CCD 对光生电荷的存储功能。 电荷转移是 CCD 的本征特点,本质是移动存储电荷的电势阱,在各电极上施加不同
西安电子科技大学硕士学位论文
24
电压形成不同电势阱,沿电势阱转移信号电荷的部分也称为转移沟道。电荷检测是指 在输出过程中对转移到 CCD 输出单元的光生电荷进行电荷的线性转换。 (2)CMOS 图像传感器 CMOS(ComplementaryMetalOxideSemiconductor)的中文名字是互补金属氧化 物半导体,CMOS 图像传感器应当看做是一个图像采集传输系统。典型的商业化的 CMOS 图像传感器结构包含:一个光敏像素阵列核心单元,它把多行电平信号传输到 一个帧的输出;时序逻辑操作寄存器、时钟控制、增益调节、积分时间、图像格式寄 存器及片内的可编程功能寄存器;同时存在着模数转换器。CMOS 传感器包括像素单 元阵列、逻辑寄存器、锁相环、时钟脉冲发生器和模数转换电路在内的全部模块。与 传统的 CCD 对比,CMOS 传感器的单一图像系统集成的特点降低了功耗,节省了芯 片空间,简化了外围的驱动配置电路,降低了开发的难度以及总体价格。 CMOS 图像传感器在光生电荷读出方式上与 CCD 采用的原理存在显著不同, CMOS 在每个像素电路中内置电荷放大电路,采用这种电荷转换电路节约了信号传输 的带宽资源。随着其像素单元制造工艺的提高,例如采用电子开关提升开断速度、选 用互阻放大器、增加双采样保持电路降低固定图形噪声。使得 CMOS 图像传感器相 对 CCD 的性价比越来越高
\subsection{数据传输接口选型}
IMX135 的图像输出采用 MIPI CSI-2 接口,下面对该接口协议进行介绍。MIPI 英文全称为 MobileIndustryProcessorInterface,即移动产业处理接口,该联盟联合各 大科技公司共同制定了一套接口通信标准规范,把移动设备系统的外设如摄像头、显 示屏等接口进行标准化,从而增加各个系统间的设计灵活性,降低开发的难度和成本, 提升系统的抗干扰能力[33][34]。其中关于摄像头的接口协议为 CSI-2,即第二代 Camera SerialInterface,它是一种串行的高速数字图像传输接口,CSI-2 协议接口标准规定了 发送器和接收器两部分,发送器和接收器之间包括了一对时钟差分接口和 2~4 对差 分数据接口,如下图是一个以 I2C 标准作为控制的 CSI-2 的接口示意图。

\subsection{图像处理显示平台功能选型}

从上表中我们可以清楚的看到树莓派板的硬件系统的计算能力和丰富的外设扩 展能力,可将多种嵌入式系统外设接入其中,树莓派板价格较低,并且支持多种 Linux 内核的第三方操作系统,开发简单,非常适合嵌入到产品中做二次开发,并且博通 BCM2386 系列处理器兼容性好,可将其核心 SOC 升级成高版本系列,而无需对软件 做更多的更改设计,树莓派将软件和操作系统放在 SD 卡上,极大的便利了用户对操 作系统系统及应用软件的升级换代。树莓派的实物图如图 4.2 所示。
