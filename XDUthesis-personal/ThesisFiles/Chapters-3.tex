\chapter{显微图像处理平台系统设计}
本设计中采用了树莓派板 ARM 处理器,运用板中的 CSI-2 串行接口,将 IMX135 相机采集的显微图像传输至 BCM2836 处理器,在图像处理器 GPU 中进行数据的编 解码,对图像进行降噪、图像增强等处理,并以 H.264 格式的视频格式存储在 SD 卡 上,并通过 HDMI 接
口在显示器上播放显示,采用基于误差修正的算法对图像进行 缺损检测。树莓派板的硬件框图系统如图 4.1 所示。
\section{树莓派平台简介}
显微处理系统的主控核心采用开源的树莓派硬件系统,树莓派是一款基于开源的 Linux 操作系统的单板式计算机,其电路板尺寸只有一张银行卡大小,它起源于英国 的树莓派基金会,项目的发起者厄普顿其本想利用廉价硬件和自由软件激发在校学生 的计算机编程能力。第一版树莓派于 2012 年由英国剑桥大学正式向外界发布,现已 经发行至最新的第二代 B+型号

\section{图像传感器接口驱动程序}
\subsection{接口驱动简介}
V4L2 是 Linux 操作系统下摄像头驱动开发的协议,在深入分析了 V4L2API 中的 数据结构及 V4L2 驱动框架后,利用 V4L2 的框架进行了 CSI-2 摄像头的驱动设计。 
\subsection{V4L2驱动框架协议了解使用}
V4L2 中包含一整套的数据结构和驱动回调函数[39],下面对其进行了介绍。 (1)在 Linux 操作系统内核中,系统目录中的 linux/videodev2.h 函数封装库中 包括了开发过程中常用的数据结构体。如下列出常用的结构体并加以说明。


\section{显微图像预处理}
\subsection{预处理图像算法研究分析}
均值
中值
高斯
。。。
\section{实时图像处理算法分析}
融合算法
阈值分割算法

\section{上位机图像显示处理}


