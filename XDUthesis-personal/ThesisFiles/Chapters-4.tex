\chapter{显微成像系统的检测实验}

通过搭建实验检测平台,对光学薄膜的表面缺陷通过进行观测检测验证。对照本 显微成像检测系统的原理框图搭建实验平台,通过显微成像实验平台,实验完成对照 明光源的位置设置、显微图像采集格式的设置、树莓派接口通信的调试,以及对图像 数据采集传输与控制进行测试,验证了光学薄膜表面缺陷检测系统具备可行性。

\section{薄膜表面缺陷检测测试}

\subsection{实验仪器介绍和安装}
巴拉巴拉拔凉拔凉

安装

\subparagraph{硬件环境搭建}
首先组装实验装置平台,把设计的显微镜头和设计的 CMOS 驱动板通过 CS 口连 接起来,CMOS 驱动板的数据输出采用 FFC 软排线连接,组装好后的镜头如图 5.1 所示。
显微系统的照明光源采用环形白色 LED 的反射式照明,根据外界不同的环境光 照,可以调节照明 LED 灯的亮度,以适应不同透明度的薄膜的照明要求。采集不同 光照强度下的显微图像,选取具有最佳对比度的图像作为最终需要识别处理的图像。 
\subparagraph{软件系统设置}

搭建的实验系统平台如图 5.2 所示,连接显微镜头和 CMOS 采集板,采用反射式 的照明方式,打开上位机软件界面,并调节好显微镜焦距,观测镜片表面的薄膜,如 图 5.2 所示。
%\subparagraph{使用方式解释说明}

\subsection{缺陷检测结果分析}
在显微成像系统测试中,选取了实验室最常用的光学透镜的表面薄膜,观测了增 透膜和增反膜表面的缺陷程度,对表面出现的缺陷进行了分析。调整不同的光照方式, 对同一片增透膜进行了多次检测,以消除光照对检测误差的影响,采集到不同的薄膜 显微图像,并对传输上来的图片进行了图像预处理,采用均值滤波算法滤除高斯噪声, 并对图像进行二值化处理,对采集的多幅图像进行比较选择,排除对焦不同等差异条 件。采集到的薄膜表面的显微图像原始数据。
在上位机上对显微图像采用中值滤波进行图像降噪预处理,处理 结果如图 5.4 所示。
采用阈值分割处理算法对预处理过的图像做缺损检测处理,选定标准完整的薄膜 表面图像作为标准背景图,将显微系统采集到的带有缺损的薄膜表面图像与之对比, 对缺陷图像进行阈值分割处理,并截取放大显示识别出的典型的缺陷形状,选取记录 结果如下所示。

对薄膜的缺陷检测实验中,CMOS 相机采集的速度在 1080P 格式下达到每秒 30 帧,通过 CSI 接口传到显示器,实现了对薄膜的实时检测和识别,还能够通过网口传 到局域网实现远程缺陷检测。相比传统的缺陷检测系统,整个缺陷检测系统硬件上采 用了最新的 ARM 处理器,选用最新的 CMOS 图像传感器 IMX135,并通过 CSI-2 串 行接口输出图像,实现了高速实时的检测特点。整个检测系统结构紧凑,Raspbian 操 作系统可更新升级,实现了微型化智能化的检测特点。
\subsection{本章小节}
本章搭建了基于树莓派处理器的显微成像系统平台,选用不同光照强度的光源进 行照明,测试了实验室的光学薄膜表面的缺陷,验证了该系统的实时检测性能,并对 采集的薄膜表面的图像进行处理,识别出缺陷的类型和形状,达到了系统的总体设计要求。
