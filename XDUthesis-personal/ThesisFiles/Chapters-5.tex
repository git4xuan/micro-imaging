\chapter{总结与展望}
\section{论文工作总结}
对光学薄膜表面存在的缺陷进行检测,采用传统显微检测的方法不再适应现代化 工艺的要求。本文设计出一种基于 ARM 的显微成像系统,
(1)对本文的研究原理和背景做基本的阐述,希望解决的问题。并查阅相关显微成像和光学薄膜的国内外发展现状,在此基础上进一步改进,介绍本文的主要结构和目前安排。

(2)设计显微成像系统硬件部分,对显微成像的光学原理做出阐述,并进行目镜物镜的选型。图像传感器功能和类型性能比较选型,数据传输接口的比较,显示处理方式的比较和选择,设计出一个较完整的低耦合化方案。

(3)对显微图像处理平台进行系统设计,查阅研究了相关CSI的接口在树莓派的驱动程序,介绍V4L2框架的应用和使用。并分析图像算法的在预处理和处理上的研究,包括线性滤波、中值滤波、高斯滤波,边缘检测等方法,并研究了显微图像的相关拼接算法,设计了阈值分割和像素投影算法。最后在Web上展示处理后的图像,同时可以查阅前期归档后的图像。

(4)设计显微成像的实验测试系统,验证系统可行性,并依据照明光线变量验证可以达到的系统精度,满足系统的测试要求指标。


\section{未来发展展望}
本显微成像对薄膜的检测系统还处于初级研究阶段,但本显微成像检测系统还存 在着诸多的不足,未来还需对系统进行进一步的深入研究。改进方向体现在如下几点: 
(1)在显微成像的 CMOS 相机方面,对图像的采集速度还有一定提升的空间, 后期还可以将帧频提升至 60 帧每秒,以提高实时检测方面的精度。对相机镜头的加工工艺也有进一步提升的空间。
(2)在对图像处理进行算法分析方面,文中的图像的滤波插值算法和融合分割方法虽然取得了预期效果,但是还存在进一步提升的空间,可以进一步提升算法的智能化,加强数据的处理和算法的效率。
(3)图像web显示界面还有一定的优化空间,包括后期的数据归档和存储都可以改进,如定期上传不常用数据到云中归档,定期调用数据进行数据迁移。
(4)在显微图像缺陷的模式识别方面,对光学薄膜表面缺陷图像模式处理的智能化识别还需要进一步研究,可以针对图像缺陷的生成机制进行深入研究,对不同类型的缺陷进行更好优化显示和识别,借此提高该检测系统对光学薄膜缺陷图像的识别率。 



