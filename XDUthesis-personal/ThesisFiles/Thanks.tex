\begin{thanksfor}
	
转眼又到毕业季了,我的研究生学习生活也将结束,现在即将完成硕士论文的撰 写,能把读研的两年学习和积累最终完成一篇硕士毕业论文,衷心的感谢实验室的各 位师长和同学们的热情帮助! 首先感谢我的导师邵晓鹏教授,邵老师治学严谨、工作勤勉、对学术精益求精, 在我攻读硕士学位期间,无论在学习中还是生活上均给予了我极大的帮助和关怀,尤 其在我工作遇到困难的时候及时地提醒我,让我重新认识自己,更好的完成工作。 感谢王琳老师,在上研期间参与的项目中,王老师亲自为我们指导,每当项目遇 到难题时,王老师都会给予我精心指点和耐心教诲。 感谢赵小明老师,这篇论文的选题、方案的确定、研究工作的开展以及论文后期 修改,都是在赵老师的悉心指导下完成的。赵老师学识渊博、态度温和,经常百忙之 中抽时间为我们解决问题,在许多硬件技术的问题都有赖于他的指导解答。 感谢实验室的所有的老师们,师兄师姐们,从进入实验室开始,就承蒙你们的照 顾,之后相处的时间里,你们更是不遗余力地帮助我。感谢实验室同届的伙伴和师弟 师妹们,在生活学习工作中各方面的互相照应,大家一起学习、一起做项目,和你们 在一起真的很愉快! 感谢父母的养育之恩,永远记得父亲的谆谆教导,虽然您再不能亲自教导我去做 人,但我将会对您的感恩牢记于心,是我今后学习工作的不竭动力。母亲在我求学过 程中给予我无限的督促和支持。感谢我的亲人、朋友等在生活上给予我的关心和支持, 感谢你们所有的付出。 最后,感恩那些帮助过我的所有人,在这里我满怀感恩之心,祝福大家一切幸福 安康!
虽为致谢环境,其实就是一个Chapter,为啥这么费事?因为,致谢一章没有编号。直接使用 \verb=\chapter*{}=的话,页眉又不符合工作手册要求,而且要往目录中添加该章节,还需要添加两行代码;为了简单快捷的设计出符合要求,又方便用户使用,只能借 \verb=\backmatter= 模式和本模板自定义的 \verb=\comtinuematter= 模式配合环境来做了。
\end{thanksfor}